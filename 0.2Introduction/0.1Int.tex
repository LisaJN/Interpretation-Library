\section*{Preface}
\addcontentsline{toc}{section}{Preface}
The following document is a collection of figures representing interpretations of a broad range of geological structures and depositional features. The material is divided into two main branches: Earth and Mars. The collected data in the Earth section is based on multiple different scientific works published over a long period of time. Furthermore, this section contains data acquisition methods, including GPR, seismic reflection profiles, and direct observations from outcrops, where the GPR sections dominate as it is the main focus of the data library. Oppositely, the interpretations presented in the Mars section are based on the data collected by RIMFAX and RoPeR from 2022 to 2025. 

The figures in both sections are further categorised based on the dominant depositional environment or system, such as fluvial, eolian, lacustrine, or igneous settings. These categories help contextualise the observed geometries and reflection patterns, allowing for comparison across scales and sedimentary processes. Each section contains a table of all keywords used within the section to make it easier for the user to identify specific structures. 

The data library itself serves as a visual reference for recognising key facies characteristics, such as continuity, amplitude, geometry, and terminations, across different depositional contexts. While the classification aims to be representative, some overlap between systems is expected due to the complexity of natural environments. 

The data library is set to contain only keywords and figures to be as simple and easy to use as possible. The different examples are all referenced to make it easy for the user to locate the source if additional information is needed. The figures are all titled based on the work they were gathered from, where many of them are collections of structures from multiple systems. This means a figure titled "beach ridge" could contain additional examples of eolian deposits. The seismic and outcrop chapters both have a table of the different keywords used to define the examples. The GPR chapter has one table per environment, as it is more extensive compared to the other two. If the data library is expanded upon in the future, this could be applied to the other chapters as well. 
